%                                             -*- coding: utf-8 -*-
% Using LaTeX is highly recommended. 
% If you correct/upgrade anything please send it back to help others' work.

\documentclass[a4paper,oneside]{article}
\usepackage[margin=3cm]{geometry}
% =================================================================
\usepackage[english]{babel}
\selectlanguage{english}

%=================================================================
% Font encoding
% The T1 font encoding is an 8-bit encoding and uses fonts that have 256 glyphs.
\usepackage[T1]{fontenc}
\usepackage[utf8]{inputenc}
\usepackage{multirow} 
%================================================================
% Use Times New Roman
\usepackage{times}

%================================================================
% Figures
% usage: \includegraphics[width=<width>]{fig.png}
\usepackage{graphicx}

% images root folder
\graphicspath{{./figs/}}

%================================================================
% Package to create pdf hyperlinks
%------------------------------------
% Hyperref should be the last imported (except some problematic packages, e.g. algorithm)
\usepackage[colorlinks=true]{hyperref}


%%%%%%%%%%%%%%%%%%%%%%%%%%%%%%%%%%%%%%%%%%%%%%%%%%%%%%%%%%%%%%%%%%%
% HERE IS THE START OF THE DOCUMENT
%%%%%%%%%%%%%%%%%%%%%%%%%%%%%%%%%%%%%%%%%%%%%%%%%%%%%%%%%%%%%%%%%%
\begin{document}
\input{macros} % Import macros
\markright{Your Name (NEPTUN)} % one sided title page!!!
%--------------------------------------------------------------------
% title page
%--------------------------------------------------------------------

\begin{titlepage}
%bme logo 
 \begin{figure}[h]
    \centering
      \includegraphics[width=12cm]{bme_logo.pdf}
  \label{fig:bme_logo}
  \end{figure}
  \thispagestyle{empty}
  
  % generate title
  \projectlaboratorytitle
 
  \projectlaboratoryauthor{Your Name}{NEPTUN}{Specialization}{user@example.com}{Supervisor Name, PhD}{supervisor@tmit.bme.hu}{Supervisor2 Name}{supervisor2@tmit.bme.hu}
 
 
  %\tasktitle
  \tasktitle{Title of the project (one line)} 

  %\taskdescription
  \taskdescription{Short description of the project.}

  % semester Arguments: #1=Semester (format: xxxx/xx , #2=I/II (without dot!))
  \begin{center}
    \semester{2018/19}{II}
  \end{center}
  
\end{titlepage} 

%==================================================================
\begin{center}
  \section{Theory and Background}
  \label{sec:theory_background}
\end{center}

\subsection{Introduction}
\label{sec:introduction}
% TODO: Write content here.
% - Introduce Kubernetes as the de-facto standard for container orchestration.
% - Explain the challenge of network performance isolation in shared environments.
% - State the thesis of your paper.
% - Briefly outline the structure of the paper.

\subsection{The Challenge of Multi-Tenant Networking in Kubernetes}
\label{sec:challenges}
% TODO: Write content here.
% - Explain The Kubernetes Networking Model (Pods, Services, CNI).
% - Describe the "Noisy Neighbor" Problem. Use diagrams.
% - Discuss Quality of Service (QoS) in Networking (latency, jitter, bandwidth).

\subsection{Enabling Technologies}
\label{sec:enabling_technologies}
% TODO: Write content here.
% - Introduce eBPF and Cilium, highlighting the eBPF-based BandwidthManager.
% - Introduce Control Theory Fundamentals (closed-loop, Proportional Controller, "cut fast, raise slow").

\newpage
%==================================================================
\begin{center}
  \section{System Design, Implementation, and Evaluation}
  \label{sec:work}
\end{center}

\subsection{System Architecture}
\label{sec:architecture}
% TODO: Write content here.
% - Present a high-level architecture diagram showing all components and the control loop.

\subsection{The Adaptive Jitter Controller}
\label{sec:controller_implementation}
% TODO: Write content here.
% - Detail the Real-time Jitter Measurement (NetworkProbe).
% - Describe The Control Algorithm (BandwidthController).
% - Explain the Bandwidth Throttling Mechanism (patching the annotation).

\subsection{Evaluation and Results}
\label{sec:evaluation}
% TODO: Write content here.
% - Describe the Testbed Environment (workloads, iperf3).
% - Present the Verification Scenarios and Results (graphs of jitter and bandwidth).
% - Include a Discussion analyzing the results.

\subsection{Conclusion and Future Work}
\label{sec:conclusion_future_work}
% TODO: Write content here.
% - Summarize the project in the Conclusion.
% - Provide ideas for Future Work.

\newpage
%==================================================================
\section{References}
\label{sec:references}

\begin{references}{9}
\bibitem{eco} Umberto Eco, \emph{Hogyan írjunk szakdolgozatot?},
  Kairosz Kiadó, 2000, ISBN: 9639137537.

\bibitem{esterhazy} Esterházy Péter, \emph{Termelési-regény (Kisssregény),}
  Magvető Könyvkiadó, 2004, ISBN: 9631423948.

\bibitem{web} \emph{Tájékoztató a Műszaki Informatika Szak önálló
    laboratórium tantárgyainak 2008/9. tanév I. félévi lezárásáról a
    BME TMIT-en (VITMA367, VITMA380, VITT4353, VITT4330),}
  \url{http://inflab.tmit.bme.hu/08o/lezar.shtml}, szerk.: Németh Felicián,
  2008. november 5.

\bibitem{wikipedia} Wikipedia contributors, \emph{Wikipedia:Academic
    use}, Wikipedia, The Free Encyclopedia, 2011 Nov 11.  Available
  from: \\ \url{http://en.wikipedia.org/w/index.php?title=Wikipedia:Academic\_use\&oldid=460041928}

\end{references}

\subsection{Attached documents}
\label{sec:attached_documents}

Source code etc.

\end{document} 

%%% Local Variables: 
%%% mode: latex 
%%% TeX-master: t 
%%% End:

