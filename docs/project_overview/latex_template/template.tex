%                                             -*- coding: utf-8 -*-
% Using LaTeX is highly recommended. 
% If you correct/upgrade anything please send it back to help others' work.

\documentclass[a4paper,oneside]{article}
\usepackage[margin=3cm]{geometry}
% =================================================================
\usepackage[english]{babel}
\selectlanguage{english}

%=================================================================
% Font encoding
% The T1 font encoding is an 8-bit encoding and uses fonts that have 256 glyphs.
\usepackage[T1]{fontenc}
\usepackage[utf8]{inputenc}
\usepackage{multirow} 
%================================================================
% Use Times New Roman
\usepackage{times}

%================================================================
% Figures
% usage: \includegraphics[width=<width>]{fig.png}
\usepackage{graphicx}

% images root folder
\graphicspath{{./figs/}}

%================================================================
% Package to create pdf hyperlinks
%------------------------------------
% Hyperref should be the last imported (except some problematic packages, e.g. algorithm)
\usepackage[colorlinks=true]{hyperref}


%%%%%%%%%%%%%%%%%%%%%%%%%%%%%%%%%%%%%%%%%%%%%%%%%%%%%%%%%%%%%%%%%%%
% HERE IS THE START OF THE DOCUMENT
%%%%%%%%%%%%%%%%%%%%%%%%%%%%%%%%%%%%%%%%%%%%%%%%%%%%%%%%%%%%%%%%%%
\begin{document}
\input{macros} % Import macros
\markright{Norbert Bendegúz Hasznos (DN04PZ)} % one sided title page!!!
%--------------------------------------------------------------------
% title page
%--------------------------------------------------------------------

\begin{titlepage}
%bme logo 
 \begin{figure}[h]
    \centering
      \includegraphics[width=12cm]{bme_logo.pdf}
  \label{fig:bme_logo}
  \end{figure}
  \thispagestyle{empty}
  
  % generate title
  \projectlaboratorytitle
 
  \projectlaboratoryauthor{Norbert Bendegúz Hasznos}{DN04PZ}{Artificial Intelligence and Data Science}{bendeguz.hasznos@gmail.com}{Markosz Maliosz, PhD}{supervisor@tmit.bme.hu}
 
 
  %\tasktitle
  \tasktitle{Adaptive Bandwidth Controller in Kubernetes} 

  %\taskdescription
  \taskdescription{Short description of the project.}

  % semester Arguments: #1=Semester (format: xxxx/xx , #2=I/II (without dot!))
  \begin{center}
    \semester{2025/2026}{I}
  \end{center}
  
\end{titlepage} 

%==================================================================
\begin{center}
  \section{Theory and Background}
  \label{sec:theory_background}
\end{center}

\subsection{Introduction}
\label{sec:introduction}
% TODO: Write content here.
% - Introduce Kubernetes as the de-facto standard for container orchestration.
Kubernetes (K8s) is an open-source system for automating deployment, scaling, and managing container-
ized applications. It was historically "heavy" and used for massive cloud datacenters; however, trends
show. K8s is becoming widely adopted now in edge computing and smaller-scale environments, too. The reason behind this adoption is the flexibility, self-healing, scalability and a wide range of features
it provides right out of the box. However, using K8S for Edge computing introduces some challenges, like
network performance isolation. In cloud datacenters, if a video download slows down a background
update, nobody will notice that, but in industrial settings, where real-time determinism is required, this
"noisy neighbour" problem will be dangerous. Edge nodes often run heterogeneous workloads, which
means safety-critical applications - like robot control loops - will share the same physical network interface
with low-priority applications - like telemetry logging. Kubernetes, by default, does not solve this issue; it
provides only basic Quality of Service (QoS) classes based on resource requests and limits defined in Pod
specifications. Recent research by Yakubov and Hästbacka (2025) \cite{yakubov2025} found that while distributions like
K3s are great for saving RAM (Random Access Memory); standard Kubernetes (via kubeadm) actually
has superior data plane throughput and latency stability under load. This comes from the fact that kubeadm
does not compromise the kernel-level networking stack to save binary size. This thesis solves the project
goal of providing a "Hybrid Deterministic Network Controller". Instead of switching to a lightweight
Kubernetes distribution, we build upon the standard Kubernetes networking stack. We solved the noisy
neighbour problem by adding a custom closed-loop controller: a Python-based one that actively probes the
network for congestion, and uses Cilium eBPF to dynamically throttle non-critical traffic. This provides
that the safety-critical applications have a fast and deterministic lane, even when the network is under stress.
% - Briefly outline the structure of the paper.

\subsection{The Challenge of Multi-Tenant Networking in Kubernetes}
\label{sec:challenges}
% TODO: Write content here.
% - Explain The Kubernetes Networking Model (Pods, Services, CNI).
The Kubernetes networking model fundamentally separates the application layer from the infrastructure
layer. While this enables fast scaling, it introduces significant opacity in how the physical network resources
are utilised.

\subsection{Kubernetes Networking Model}
\label{sec:kubernetes_networking_model}
The Kubernetes networking model establishes a single network where every pod can communicate directly with every other pod, without the use of Network Address Translation (NAT) \cite{k8sdocs}. This is enabled via the Container Network Interface (CNI), which connects Kubernetes orchestration to the Linux networking stack.
In a standard deployment (e.g, using default CNI plugins like Flannel or Calico), without additional configuration and tuning, the Linux kernel treats all packet flows equally. The packet scheduler
in the kernel won’t differentiate between critical traffic (e.g., control loops for industrial robots) and non-
critical traffic (e.g., telemetry data uploads). The kernel relies on basic queuing mechanisms like First 
In First Out (FIFO) or CoDel (Controlled Delay), optimize for general performance, rather than strict isochronous delivery. As noted by Solber et al. (2024)\cite{solber2024}, this "best-effort" approach is sufficient for web services but introduces unacceptable non-determinism for industrial control loops.

\subsection{The Noisy Neighbour Problem}
\label{sec:noisy_neighbour_problem}
The Kubernetes networking model enforces a "flat" network space where every pod can communicate with
every other pod without NAT (Network Address Translation) [4]. This is achieved through the Container.
Network Interface (CNI), which acts as the translation layer between the K8S orchestration and the under-
lying Linux networking stack. In a standard deployment (e.g., using default CNI plugins like Flannel or
Calico), without additional configuration or tuning, the Linux kernel treats all packet flows equally. The
packet scheduler in the kernel won’t differentiate between critical traffic (e.g., control loops for industrial
robots) and non-critical traffic (e.g., telemetry data uploads). The kernel relies on basic queuing mechanisms, like FIFO (First-In-First-Out. In First Out (FIFO) or CoDel (Controlled Delay), optimise for general
performance, rather than strict isochronous. delivery. As noted by Solber et al. (2024) \cite{solber2024}, this "best-effort" approach is sufficient for web services, but introduces unacceptable non-determinism for industrial control
loops.

\subsection{Quality of Service (QoS) Misconceptions}

\label{sec:qos_misconceptions}

The critical problem and limitation in default K8s is the scope of its Quality of Service (QoS) classes. While K8s provides Guaranteed, Burstable, and BestEffort classes, these only govern
CPU cycles and memory pages \cite{k8sdocs}, therefore, they do not natively isolate Network I/O or bandwidth.

For an industrial edge system, the "Network Health" is not a single metric, but a combination of constraints:
\begin{itemize}

  \item Latency(ms): Time to travel from source to destination.

  \item Packet Loss(percentage): Percentage of packets that fail to reach their destination.

  \item Jitter(PDV)(ms): Variability in packet latency over time.

  \item Throughput(Mbps): The rate of successful message delivery over a communication channel.

\end{itemize}
Standard Linux networking creates a conflict: maximizing throughput degrades jitter. Our architecture aims to solve this conflict by implementing network-aware QoS,
which dynamically adjusts bandwidth allocations based on real-time network conditions. 


\subsection{Enabling Technologies}
\label{sec:enabling_technologies}
Extended Berkeley Packet Filter (eBPF) is a kernel-level technology that enables developers to run sand-
boxed programs in the kernel without modifying the kernel source code or loading additional modules. \cite{cilium}
In the Kubernetes ecosystem, the Cilium CNI leverages eBPF to bypass the limitations imposed by
legacy iptables rules, which are known to create bottlenecks in container networking.
Our system is built on the Cilium Bandwidth Manager. Unlike traditional Linux traffic control mechanisms-
anisms, Cilium implements an Earliest Departure Time (EDT) model directly at the Traffic Control (TC)
egress hook. \cite{cilium} Instead of buffering packets in large queues (which leads to latency spikes). Something in here about something 

\newpage
%==================================================================
\begin{center}
  \section{System Design, Implementation, and Evaluation}
  \label{sec:work}
\end{center}

In this section, we talk about the design, and implementation of our controller, and the evaluation of its performance.

\subsection{System Architecture}
\label{sec:architecture}


\subsection{The Adaptive Jitter Controller}
\label{sec:controller_implementation}
% TODO: Write content here.
% - Detail the Real-time Jitter Measurement (NetworkProbe).
% - Describe The Control Algorithm (BandwidthController).
% - Explain the Bandwidth Throttling Mechanism (patching the annotation).

\subsection{Evaluation and Results}
\label{sec:evaluation}
% TODO: Write content here.
% - Describe the Testbed Environment (workloads, iperf3).
% - Present the Verification Scenarios and Results (graphs of jitter and bandwidth).
% - Include a Discussion analyzing the results.

\subsection{Conclusion and Future Work}
\label{sec:conclusion_future_work}
% TODO: Write content here.
% - Summarize the project in the Conclusion.
% - Provide ideas for Future Work.

\newpage
%==================================================================
\section{References}
\label{sec:references}

\begin{references}{9}

\bibitem{yakubov2025} Diyaz Yakubov and David Hästbacka, \emph{Comparative Analysis of Lightweight Kubernetes Distributions for Edge Computing: Performance and Resource Efficiency}, 
  IEEE European Symposium on Service-Oriented and Cloud Computing (ESOCC), 2025. 
  (Also available as arXiv:2504.03656).

\bibitem{solber2024} D. Solber, et al., \emph{Enhancing IIoT Infrastructures with Kubernetes: Advanced Edge Cluster Management},
  IEEE International Conference on Emerging Technologies and Factory Automation (ETFA), 2024.

\bibitem{cilium} Isovalent / The Linux Foundation, \emph{Cilium Documentation: eBPF-based Networking, Security, and Observability},
  \url{https://docs.cilium.io/en/stable/overview/intro/}, 
  accessed: 2025.

\bibitem{k8sdocs} The Kubernetes Authors, \emph{Kubernetes Documentation: Production Environment Container Runtimes}, 
  The Linux Foundation, 2024. 
  Available from: \\ \url{https://kubernetes.io/docs/setup/production-environment/}

\end{references}

\subsection{Attached documents}
\label{sec:attached_documents}

Source code etc.

\end{document} 

